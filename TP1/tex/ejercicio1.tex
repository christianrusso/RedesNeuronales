\section{Ejercicio: Cancer de mamas}
Los datos son resultado de un examen que se realizan para diagnosticar cancer de mamas. Para cada paciente contamos con 10 caracteristicas de las imagenes de las celulas, entre estas caracteristicas contamos con diagnostico, radio, perimetro, textura, area, suavidad, etc. 

Cada una de estas entradas corrresponde a los datos de distintos pacientes, es importante remarcar que se cuenta con el atributo del diagnostico final quen indica si un tumor es benigno o maligno (0 y 1 respectivamente).

Sobre este primer dataset se buscara entrenar una red neuronal para poder predecir, dado un nuevo paciente con sus datos, si su estado sera el de tener cancer o no.

Este tipo de problemas se denomina un \textbf{problema de clasificacion}
