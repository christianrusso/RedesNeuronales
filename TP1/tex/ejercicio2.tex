\section{Ejercicio: Eficiencia energetica}

En el siguiente problema contamos con los datos sobre la carga energetica para calefaccionar y refrigerar edificios en funcion de ciertas caractersisticas de los mismos.

El analisis se realizo tomando en cuenta edificios de diversas caracteristifcas, que difieren con respecto a la superficie y distribucion de las areas de reflejo, la orientacioon, etc.

Para cada edificio contamos con area de la superficie total, area de las paredes, area del techo, orientacion, area de reflejo total, etc.

Es importante remarcar que tambien contamos, para cada edifico del dataset, la cantidad de energio necersaria para realizar una calefaccion y refrigeracion adecuada.

Sobre este segundo dataset se buscara entrenar una red neuronal para poder aproximar la funcion que determina la cantidad de energia necesaria para
la calefaccion y refrigeracion de un edificio en funcion de los parametros de entrada. Es decir, es un problema de regresion.

\emph{\color{red} FALTA AGREGAR DETALLES DE NUESTRA IMPLEMENTACION}