\section{Entregable}
\subsection{Contenido del entregable}

Dentro del archivo comprimido entregado se encuentra los archivos python correspondientes a la implementacion del trabajo practico. Tambien se encuentra el informe y las redes entrenadas del ejercicio 1 y del ejercicio dos (red\_ej1 y red\_ej2 respectivamente).
\subsection{Modo de ejecucion}

La manera de ejecutar el programa dado es la siguiente:


%python main.py ej1 -t './datasets/tp1_ej1_training.csv' 'red' 0.1 10 0.75  0.01 0.6 1 0 
\begin{verbatim}
$ python main.py < ejercicio > < opcion > < archivo_entrada > < archivo_red_salida >  < epsilon > 
                       < tau > < learning_rate > < momentum > < holdout_rate > < modo >
\end{verbatim}

Donde:

\begin{enumerate}
\item ejercicio: \textbf{ej1} o \textbf{ej2} dependiendo el ejercicio que se quiera ejecutar.
\item opcion: 
\begin{enumerate}
\item -t: Entrena y guarda la red.
\item -l: Carga la red ya entrenada y prueba el nuevo dataset.
\end{enumerate}
\item archivo\_entrada: archivo de dataset.
\item archivo\_red\_salida: archivo donde se guardara la red.
\item epsilon: Error maximo permitido.
\item tau: Numero maximo de epocas para ejecutar.
\item learning\_rate: Coeficiente de aprendizaje.
\item momentum: Momentum
\item holdout\_rate: Porcentaje del dataset con el que se va a entrenar.
\item modo: 
\begin{enumerate}
\item 1: Incremental
\item 0: Batch
\end{enumerate}
\end{enumerate}


Por ejemplo: 

Si queremos ejecutar el ejercicio 1 entrenandolo con el dataset \textbf{tp1\_ej1\_training.csv}, guardando la red en el archivo 
\textbf{red} con un epsilon de \textbf{0.1}, tau de \textbf{1000}, holdout\_rate de \textbf{0.75} learning\_rate de \textbf{0.01}, momentum de \textbf{0.6} con el modo \textbf{Incremental} seria:

\begin{verbatim}
$ python main.py ej1 -t 'tp1_ej1_training.csv' 'red' 0.1 1000 0.75  0.01 0.6 1 
\end{verbatim}

En caso de querer cargar una red ya guardada seria, por ejemplo:

\begin{verbatim}
$ python main.py ej1 -l 'red' 'tp1_ej1_training.csv'
\end{verbatim}

Siendo \textbf{red} la red guardada y \textbf{tp1\_ej1\_training.csv} con el que va a probar.