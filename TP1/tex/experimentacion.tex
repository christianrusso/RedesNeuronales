\section{Experimentacion}
\subsection{Busqueda exhautiva}
Para la realizacion de los experimientos implementamos un algoritmo de busqueda exhautiva para buscar el mejor modelo en cada caso. 

El algoritmo es el siguiente:

\begin{lstlisting}[caption=grid\_search]
def grid_search(param_grid):
	grid = ParameterGrid(param_grid)
	best_error = None
	best_params = None
	print "Ejercicio "+str(param_grid["1"])
	print len(grid), "modelos distintos"
	i = 0
	print "i 	error(training, validacion)"
	start = timer()
	for params in grid:
	    e = train(params['1'], params['2'], params['3'],params['4'],params['5'],params['6'],params['7'],
	    	params['8'],params['9'],params['10'],params['11'], True )
	    print i, "	", e
	    i += 1
	    if i%50 == 0:
	    	end = timer()
	    	print int((end-start)/60), " min"
	    if best_error == None or e < best_error:
	    	best_error = e
	    	best_params = [params['1'], params['2'], params['3'],params['4'],params['5'],params['6'],params['7'],params['8'],
	    	params['9'],params['10'],params['11']]
	print "MEJOR ERROR Y MODELO EJERCICIO "+str(param_grid["1"])
	print "training, validacion:", best_error
	print "parametros:", best_params
	end = timer()
	print int((end-start)/60), " min"
\end{lstlisting}

Y la forma de instanciarlo es la siguiente:

\begin{lstlisting}[caption=Instanciacion]
param_grid = {'1': [1],"2":['./datasets/tp1_ej1_training.csv'],"3": [None], "4":[0.1], "5": [200], 
			"6":[0.001,0.01,0.1,0.5, 0.005, 0.2, 0.3, 0.4], "7":[0.1,0.3,0.5, 0.7, 0.9], "8": [0.70], 
			"9": [0,1], "10":[0], "11":[[5],[10],[15],[20],[25],[15,15],[5,5],[10,10]]}
\end{lstlisting}

La idea es bastante simple, lo que hacemos es pasar por parametro una lista con todos los posibles para parametros y el algoritmo corre un ciclo, donde en cada ciclo prueba con un parametro distinto guardandoce el mejor modelo para finalmente imprimirlo por la consola.

De esta forma logramos quedarnos con la mejor arquitectura en cada ejercicio. PERO DE TODAS FORMAS, HICIMOS PRUEBAS MANUALES PARA VALIDAD QUE ESTO ERA CIERTO Y PARA ESTO HICIMOS EL SIGUIENTE ANALISIS.
\subsection{Ejercicio 1}
\subsection{Ejercicio 2}
\subsection{Conclusion}
