\documentclass[a4paper, 10pt, twoside]{article}

\usepackage[top=1in, bottom=1in, left=1in, right=1in]{geometry}
\usepackage[utf8]{inputenc}
\usepackage[spanish, es-ucroman, es-noquoting, activeacute]{babel}
\usepackage{setspace}
\usepackage{fancyhdr}
\usepackage{lastpage}
\usepackage{amsmath}
\usepackage{amsfonts}
\usepackage{amsthm}
\usepackage{algpseudocode}
\usepackage[]{algorithm2e}
\usepackage{verbatim}
\usepackage{fancyvrb}
\usepackage{graphicx}
\usepackage{float}
\usepackage{enumitem} % Provee macro \setlist
\usepackage{tabularx}
\usepackage{multirow}
\usepackage{hyperref}
\usepackage{lscape}
\usepackage{xspace}
\usepackage{qtree}
\usepackage[toc, page]{appendix}


%%%%%%%%%% Constantes - Inicio %%%%%%%%%%
\newcommand{\titulo}{Trabajo Práctico 1}
\newcommand{\materia}{Redes Neuronales}
\newcommand{\integrantes}{Barijhoff - Ishikame - Russo}
\newcommand{\cuatrimestre}{Segundo Cuatrimestre de 2016}
%%%%%%%%%% Constantes - Fin %%%%%%%%%%

%%%%%%%%%% Configuración de Fancyhdr - Inicio %%%%%%%%%%
\pagestyle{fancy}
\thispagestyle{fancy}
\lhead{\titulo\ · \materia}
\rhead{\integrantes}
\renewcommand{\footrulewidth}{0.4pt}
\cfoot{\thepage /\pageref{LastPage}}

\fancypagestyle{caratula} {
   \fancyhf{}
   \cfoot{\thepage /\pageref{LastPage}}
   \renewcommand{\headrulewidth}{0pt}
   \renewcommand{\footrulewidth}{0pt}
}
%%%%%%%%%% Configuración de Fancyhdr - Fin %%%%%%%%%%


%%%%%%%%%% Miscelánea - Inicio %%%%%%%%%%
% Evita que el documento se estire verticalmente para ocupar el espacio vacío
% en cada página.
\raggedbottom

% Separación entre párrafos.
\setlength{\parskip}{0.5em}

% Separación entre elementos de listas.
\setlist{itemsep=0.5em}

% Asigna la traducción de la palabra 'Appendices'.
\renewcommand{\appendixtocname}{Apéndices}
\renewcommand{\appendixpagename}{Apéndices}

\newcommand{\diagrama}[1]{
  \begin{center}
    \includegraphics[width=16cm]{#1}
  \end{center}
}

\newcommand{\diagramadeancho}[2]{
  \begin{center}
    \includegraphics[width=#1]{#2}
  \end{center}
}

\newcommand{\riesgo}[7]{
  \underline{Riesgo {#1}:}
  \begin{itemize}   
    \item \textbf{Descripción:} {#2}
    \item \textbf{Probablidad:} {#3}
    \item \textbf{Impacto:} {#4}
    \item \textbf{Exposición:} {#5}
    \item \textbf{Mitigación:} {#6}
    \item \textbf{Plan de contingencia:} {#7}
  \end{itemize}
}

\newcommand{\escenario}[7] {
  \textit{{#1}}
  \begin{itemize}
    \item \textbf{Fuente:} {#2}
    \item \textbf{Estímulo:} {#3}
    \item \textbf{Entorno:} {#4}
    \item \textbf{Artefacto:} {#5}
    \item \textbf{Respuesta:} {#6}
    \item \textbf{Medición:} {#7}
  \end{itemize}
}

%%%%%%%%%% Miscelánea - Fin %%%%%%%%%%

\begin{document}


%%%%%%%%%%%%%%%%%%%%%%%%%%%%%%%%%%%%%%%%%%%%%%%%%%%%%%%%%%%%%%%%%%%%%%%%%%%%%%%
%% Carátula                                                                  %%
%%%%%%%%%%%%%%%%%%%%%%%%%%%%%%%%%%%%%%%%%%%%%%%%%%%%%%%%%%%%%%%%%%%%%%%%%%%%%%%


\thispagestyle{caratula}

\begin{center}

\includegraphics[height=2cm]{DC.png} 
\hfill
\includegraphics[height=2cm]{UBA.jpg} 

\vspace{2cm}

Departamento de Computación,\\
Facultad de Ciencias Exactas y Naturales,\\
Universidad de Buenos Aires

\vspace{4cm}

\begin{Huge}
\titulo
\end{Huge}

\vspace{0.5cm}

\begin{Large}
\materia
\end{Large}

\vspace{1cm}

\cuatrimestre

\vspace{4cm}

\begin{tabular}{|c|c|c|}
\hline
Apellido y Nombre & LU & E-mail\\
\hline
Barijhoff, Hernan           & 338/13 & hernanfb@live.com.ar \\
Ishikame, Emiliano               & 861/11 & emilianoishikame@yahoo.com.ar \\
Russo, Christian              & 679/10 & christian.russo8@gmail.com\\
\hline
\end{tabular}

\end{center}

\newpage

\tableofcontents

\newpage


%%%%%%%%%%%%%%%%%%%%%%%%%%%%%%%%%%%%%%%%%%%%%%%%%%%%%%%%%%%%%%%%%%%%%%%%%%%%%%%
%% Introducción                                                              %%
%%%%%%%%%%%%%%%%%%%%%%%%%%%%%%%%%%%%%%%%%%%%%%%%%%%%%%%%%%%%%%%%%%%%%%%%%%%%%%%

\section{Introducción y Teoria}
\subsection{Introducción}
En el presente trabajo utilizaremos redes neuronales artificiales entrenadas con retropropagación de errores (backpropagation) para modelar dos problemas distintos. 

La idea será entrenar la red con información contenida en dos bases de datos.

El primer problema consiste en, dada unas muestras de tumores de mamas, poder calificar si un tumor corresponde o no con tumores \textbf{malignos} o \textbf{benignos}.

El segundo problema dada una muestra con ciertos atributos de edificios, tales como superficie total u orientación, podremos predecir la \textbf{cantidad de energía necesaria para calefaccionar y refrigerar los edificios}.

Se espera entonces, que utilizando lo que vimos en la materia podamos desarrollar redes que resuelvan correctamente ambos problemas. 

Adicionalmente, se espera hacer un análisis y experimentación del funcionamiento de un perceptron multicapa con retropropagación de errores.

\subsection{Introducción Teórica}
La idea principal del perceptron multicapa es utilizar el paradigma de aprendizaje supervisado con un algoritmo de corrección de errores. Es una red neuronal artificial formada por múltiples capas, esto le permite resolver problemas que no son linealmente separables, lo cual es la principal limitación del perceptron simple.

\begin{center}
\includegraphics[width=0.25\textwidth]{img/psimple}
\end{center}

El aprendizaje supervisado se basa en un entrenamiento en el cual se provee al sistema con información de las entradas y de igual forma se proveen las salidas esperadas para cada entrada en particular.

Intuitivamente, el perceptron multicapa permite aproximar funciones, categorizar y encontrar patrones.

\begin{center}
\includegraphics{img/pmulticapa}
\includegraphics[width=0.35\textwidth]{img/pmulticapa2}
\end{center}

Las neuronas de la capa oculta usan como regla de propagacion la suma ponderada de las entradas con los pesos sinápticos $w_{ij}$ y sobre esa suma ponderada, se aplica una función de transferencia de tipo sigmoide, que es acotada en respuesta.

\newpage
Gráfico de función sigmoide:

\begin{center}
\includegraphics[width=0.30\textwidth]{img/sigmoide}
\end{center}

Sobre el aprendizaje, para éste se suele usar en este tipo de redes recibe el nombre de backpropagation. Como función de coste global, se usa el error cuadrático medio. Es decir, que dado un par ($x_k$, $d_k$) correspondiente a la entrada k de los datos de entrenamiento y salida deseada asociada se calcula la cantidad.

Por otro lado, las capas pueden clasificarse en tres tipos:

\begin{enumerate}
\item \textbf{Capa de entrada:} Constituida por aquellas neuronas que introducen los patrones de entrada en la red. 
\item \textbf{Capas ocultas:} Formada por aquellas neuronas cuyas entradas provienen de capas anteriores y cuyas salidas pasan a neuronas de capas posteriores.
\item \textbf{Capa de salida:} Neuronas cuyos valores de salida se corresponden con las salidas de toda la red.
\end{enumerate}


En general, la estructura de una neurona se puede representar de la siguiente manera:

\begin{center}
\includegraphics[width=0.6\textwidth]{img/neurona}
\end{center}

Podemos observar a los elementos más importantes, al vector de entrada de la neurona $(x_1,...,x_m)$, a los pesos correspondientes como $w_{ij}$, a la función de activacio y al elemento de salida. A partir de esta estructura básica, la neurona puede mapear las entradas para obtener a la salida una respuesta deseada que pudiera pertenecer a alguna función a de terminar.


La función de activación trata de simular el mecanismo que realiza el sistema de neuronas en el cerebro, que se basa en la exitación de las neuronas hasta un cierto punto en el cúal se pasa un umbral en el que dicha neurona dispara la información que le corresponde. 

Entre las diferentes funciones de activación podemos encontrar la sigmoide.


Por otro lado, el perceptron cuenta con un coeficiente de entrenamiento que indica que tanto varian los pesos entre iteración e iteración, por lo cúal, indica que tan lento o rápido la red se entrena.
\section{Ejercicio: Cancer de mamas}
Los datos son resultado de un examen que se realizan para diagnosticar cancer de mamas. Para cada paciente contamos con 10 caracteristicas de las imagenes de las celulas, entre estas caracteristicas contamos con diagnostico, radio, perimetro, textura, area, suavidad, etc. 

Cada una de estas entradas corrresponde a los datos de distintos pacientes, es importante remarcar que se cuenta con el atributo del diagnostico final quen indica si un tumor es benigno o maligno (0 y 1 respectivamente).

Sobre este primer dataset se buscara entrenar una red neuronal para poder predecir, dado un nuevo paciente con sus datos, si su estado sera el de tener cancer o no.

Este tipo de problemas se denomina un \textbf{problema de clasificacion}

\section{Ejercicio: Eficiencia energetica}

En el siguiente problema contamos con los datos sobre la carga energetica para calefaccionar y refrigerar edificios en funcion de ciertas caractersisticas de los mismos.

El analisis se realizo tomando en cuenta edificios de diversas caracteristifcas, que difieren con respecto a la superficie y distribucion de las areas de reflejo, la orientacioon, etc.

Para cada edificio contamos con area de la superficie total, area de las paredes, area del techo, orientacion, area de reflejo total, etc.

Es importante remarcar que tambien contamos, para cada edifico del dataset, la cantidad de energio necersaria para realizar una calefaccion y refrigeracion adecuada.

Sobre este segundo dataset se buscara entrenar una red neuronal para poder aproximar la funcion que determina la cantidad de energia necesaria para
la calefaccion y refrigeracion de un edificio en funcion de los parametros de entrada. Es decir, es un problema de regresion.


\section{Entregable}
\subsection{Contenido del entregable}

Dentro del archivo comprimido entregado se encuentra los archivos python correspondientes a la implementación del trabajo práctico. También se encuentra el informe y las redes entrenadas del ejercicio 1 y del ejercicio dos (red\_ej1 y red\_ej2 respectivamente).

Para la ejecuci\'on se necesitan las librerias matplotlib,sklearn y numpy.
En caso de no tener alguna, ejecutar por consola el siguiente comando.

\begin{verbatim}
$ sudo apt-get install python-numpy python-sklearn python-matplotlib
\end{verbatim}

\subsection{Modo de ejecución}

La manera de ejecutar el programa dado es la siguiente:

Por default, los parámetros a partir de epsilon inclusive son opcionales, se entrena con parámetros seteados.

%python main.py ej1 -t './datasets/tp1_ej1_training.csv' 'red' 0.1 10 0.75  0.01 0.6 1 0 
\begin{verbatim}
$ python main.py < ejercicio > -t < archivo_entrada > < archivo_red_salida >  < epsilon > 
                       < tau > < learning_rate > < momentum > < holdout_rate > < modo > <capas_ocultas> 
\end{verbatim}
\begin{verbatim}
$ python main.py < ejercicio > -l < archivo_entrada > < archivo_red_salida > 
\end{verbatim}

Donde:

\begin{enumerate}
\item ejercicio: \textbf{ej1} o \textbf{ej2} dependiendo el ejercicio que se quiera ejecutar.
\item opción: 
\begin{enumerate}
\item -t: Entrena y guarda la red.
\item -l: Carga la red ya entrenada y prueba el nuevo dataset.
\end{enumerate}
\item archivo\_entrada: archivo de dataset.
\item archivo\_red\_salida: archivo donde se guardara la red.
\item epsilon: Error maximo permitido.
\item tau: Numero maximo de epocas para ejecutar.
\item learning\_rate: Coeficiente de aprendizaje.
\item momentum: Momentum
\item holdout\_rate: Porcentaje del dataset con el que se va a entrenar.
\item modo: 
\begin{enumerate}
\item 1: Incremental
\item 0: Batch
\end{enumerate}
\item capas\_ocultas: Numero de unidades en cada capa oculta.
\end{enumerate}

Por ejemplo: 

Si queremos ejecutar el ejercicio 1 entrenandolo con el dataset \textbf{tp1\_ej1\_training.csv}, guardando la red en el archivo 
\textbf{red} con un epsilon de \textbf{0.1}, tau de \textbf{1000}, holdout\_rate de \textbf{0.7} learning\_rate de \textbf{0.01}, momentum de \textbf{0.9} con el modo \textbf{Incremental} y 10 neuronas en la primer capa y 2 en la segunda seria:

\begin{verbatim}
$ python main.py ej1 -t 'tp1_ej1_training.csv' ’red’ 0.1 1000 0.01 0.9 0.7 1 [10,2]
\end{verbatim}


En caso de querer cargar una red ya guardada seria, por ejemplo:

\begin{verbatim}
$ python main.py ej1 -l 'red' 'tp1_ej1_training.csv'
\end{verbatim}

Siendo \textbf{red} la red guardada y \textbf{tp1\_ej1\_training.csv} con el que va a probar.


%Pseudocodigo de Activation
\begin{center}
\noindent\fbox{
\begin{minipage}{0.5\textwidth}
\begin{algorithm}[H]
 activation($X_h$)\;
 $Y_1$ = $X_h$\;
 \For{j de 2 a L}{
    $Y_j = f_j(Y_{j-1} * w_j)$\;
  }
  return $Y_l$\;
 \caption{Activation}
\end{algorithm}
\end{minipage}
}
\end{center}

%Pseudocodigo de Correction
\begin{center}
\noindent\fbox{
\begin{minipage}{0.5\textwidth}
\begin{algorithm}[H]
 correction($Z_k$)\;
  E = (Z-$Y_l$)\;
  e = $||E||^2$\;
  \For{j de L a 2}{
    $D = E * F_j^{'} (Y_{j-1}*W_j)$\;
    $dw_j = dw_j + simbolo(Y_{j-1}^{T}*D)$\;
    $E = D*w_j$\;
  }
  return e\;
 \caption{Correction}
\end{algorithm}
\end{minipage}
}
\end{center}

%Pseudocodigo de Adaptation
\begin{center}
\noindent\fbox{
\begin{minipage}{0.5\textwidth}
\begin{algorithm}[H]
 adaptation()\;
  \For{j de 2 a L}{
    $W_j = W_j + dw_j$\;
    $dw_j = w_j + dw_j$\;
  }
 \caption{Adaptation}
\end{algorithm}
\end{minipage}
}
\end{center}

%Pseudocodigo de Trainning Batch
\begin{center}
\noindent\fbox{
\begin{minipage}{0.5\textwidth}
\begin{algorithm}[H]
 batch(X,Z)\;
  $e = 0$\;
  \For{h de 1 a P}{
    activation($X_h$)\;
    e = e + correction($Z_h$)\;
  }
  adaptation()\;
  return e\;
 \caption{Batch}
\end{algorithm}
\end{minipage}
}
\end{center}

%Pseudocodigo de Trainning Incremental
\begin{center}
\noindent\fbox{
\begin{minipage}{0.5\textwidth}
\begin{algorithm}[H]
 incremental(X,Z)\;
  $e = 0$\;
  \For{h de 1 a P}{
    activation($X_h$)\;
    e = e + correction($Z_h$)\;
    adaptation()\;
  }
  
  return e\;
 \caption{Incremental}
\end{algorithm}
\end{minipage}
}
\end{center}

%Pseudocodigo de Validacion
\begin{center}
\noindent\fbox{
\begin{minipage}{0.5\textwidth}
\begin{algorithm}[H]
 validacion?(e?,T)\;
  $e = 1$\;
  $t = 0$\;
  \While{$0 > e$ y t $<$ T}{
    e = trainning(X,Z)\;
    t = t + 1\;
  }
  return e,t\;
 \caption{validacion?}
\end{algorithm}
\end{minipage}
}
\end{center}






%Pseudocodigo de Holdout
\begin{center}
\noindent\fbox{
\begin{minipage}{0.5\textwidth}
\begin{algorithm}[H]
 holdout(e,T)\;
  $e = 1$\;
  $t = 0$\;
  v = ??\;
  \While{$e_t > \epsilon $ y t $<$ T}{
    $e_t$ = trainning($Y_{[:v]}, Z_{[:v]}$)\;
    $e_v$ = testing($X_{[:v]}, Z_{[:v]}$)\;
    t = t+1
  }
  return ??\;
 \caption{Holdout}
\end{algorithm}
\end{minipage}
}
\end{center}







\end{document}
