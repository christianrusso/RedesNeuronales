\documentclass[a4paper, 10pt, twoside]{article}

\usepackage[top=1in, bottom=1in, left=1in, right=1in]{geometry}
\usepackage[utf8]{inputenc}
\usepackage[spanish, es-ucroman, es-noquoting, activeacute]{babel}
\usepackage{setspace}
\usepackage{fancyhdr}
\usepackage{lastpage}
\usepackage{amsmath}
\usepackage{amsfonts}
\usepackage{amsthm}
\usepackage{algpseudocode}
\usepackage[]{algorithm2e}
\usepackage{verbatim}
\usepackage{fancyvrb}
\usepackage{graphicx}
\usepackage{float}
\usepackage{enumitem} % Provee macro \setlist
\usepackage{tabularx}
\usepackage{multirow}
\usepackage{hyperref}
\usepackage{lscape}
\usepackage{xspace}
\usepackage{qtree}
\usepackage[toc, page]{appendix}


% Codigo fuente.
\usepackage{listings}
\lstset{
  language=C++,
  basicstyle=\small\sffamily,
  numbers=left,
  numberstyle=\tiny,
  frame=tb,
  columns=fullflexible,
  showstringspaces=false
}

\newcommand\todo[1]{\Large\textbf{\textcolor{red}{#1}}\normalsize}


\usepackage{chngcntr} % Numeracion granular de distintos entornos.
\counterwithin{table}{section}
\AtBeginDocument{\counterwithin{lstlisting}{section}}



%%%%%%%%%% Constantes - Inicio %%%%%%%%%%
\newcommand{\titulo}{Trabajo Práctico 1}
\newcommand{\materia}{Redes Neuronales}
\newcommand{\integrantes}{Barijhoff - Ishikame - Russo}
\newcommand{\cuatrimestre}{Segundo Cuatrimestre de 2016}
%%%%%%%%%% Constantes - Fin %%%%%%%%%%

%%%%%%%%%% Configuración de Fancyhdr - Inicio %%%%%%%%%%
\pagestyle{fancy}
\thispagestyle{fancy}
\lhead{\titulo\ · \materia}
\rhead{\integrantes}
\renewcommand{\footrulewidth}{0.4pt}
\cfoot{\thepage /\pageref{LastPage}}

\fancypagestyle{caratula} {
   \fancyhf{}
   \cfoot{\thepage /\pageref{LastPage}}
   \renewcommand{\headrulewidth}{0pt}
   \renewcommand{\footrulewidth}{0pt}
}
%%%%%%%%%% Configuración de Fancyhdr - Fin %%%%%%%%%%


%%%%%%%%%% Miscelánea - Inicio %%%%%%%%%%
% Evita que el documento se estire verticalmente para ocupar el espacio vacío
% en cada página.
\raggedbottom

% Separación entre párrafos.
\setlength{\parskip}{0.5em}

% Separación entre elementos de listas.
\setlist{itemsep=0.5em}

% Asigna la traducción de la palabra 'Appendices'.
\renewcommand{\appendixtocname}{Apéndices}
\renewcommand{\appendixpagename}{Apéndices}

\newcommand{\diagrama}[1]{
  \begin{center}
    \includegraphics[width=16cm]{#1}
  \end{center}
}

\newcommand{\diagramadeancho}[2]{
  \begin{center}
    \includegraphics[width=#1]{#2}
  \end{center}
}

\newcommand{\riesgo}[7]{
  \underline{Riesgo {#1}:}
  \begin{itemize}   
    \item \textbf{Descripción:} {#2}
    \item \textbf{Probablidad:} {#3}
    \item \textbf{Impacto:} {#4}
    \item \textbf{Exposición:} {#5}
    \item \textbf{Mitigación:} {#6}
    \item \textbf{Plan de contingencia:} {#7}
  \end{itemize}
}

\newcommand{\escenario}[7] {
  \textit{{#1}}
  \begin{itemize}
    \item \textbf{Fuente:} {#2}
    \item \textbf{Estímulo:} {#3}
    \item \textbf{Entorno:} {#4}
    \item \textbf{Artefacto:} {#5}
    \item \textbf{Respuesta:} {#6}
    \item \textbf{Medición:} {#7}
  \end{itemize}
}

%%%%%%%%%% Miscelánea - Fin %%%%%%%%%%

\begin{document}


%%%%%%%%%%%%%%%%%%%%%%%%%%%%%%%%%%%%%%%%%%%%%%%%%%%%%%%%%%%%%%%%%%%%%%%%%%%%%%%
%% Carátula                                                                  %%
%%%%%%%%%%%%%%%%%%%%%%%%%%%%%%%%%%%%%%%%%%%%%%%%%%%%%%%%%%%%%%%%%%%%%%%%%%%%%%%


\thispagestyle{caratula}

\begin{center}

\includegraphics[height=2cm]{DC.png} 
\hfill
\includegraphics[height=2cm]{UBA.jpg} 

\vspace{2cm}

Departamento de Computación,\\
Facultad de Ciencias Exactas y Naturales,\\
Universidad de Buenos Aires

\vspace{4cm}

\begin{Huge}
\titulo
\end{Huge}

\vspace{0.5cm}

\begin{Large}
\materia
\end{Large}

\vspace{1cm}

\cuatrimestre

\vspace{4cm}

\begin{tabular}{|c|c|c|}
\hline
Apellido y Nombre & LU & E-mail\\
\hline
Barijhoff, Hernan           & 338/13 & hernanfb@live.com.ar \\
Ishikame, Emiliano               & 861/11 & emilianoishikame@yahoo.com.ar \\
Russo, Christian              & 679/10 & christian.russo8@gmail.com\\
\hline
\end{tabular}

\end{center}

\newpage

\tableofcontents

\newpage


%%%%%%%%%%%%%%%%%%%%%%%%%%%%%%%%%%%%%%%%%%%%%%%%%%%%%%%%%%%%%%%%%%%%%%%%%%%%%%%
%% Introducción                                                              %%
%%%%%%%%%%%%%%%%%%%%%%%%%%%%%%%%%%%%%%%%%%%%%%%%%%%%%%%%%%%%%%%%%%%%%%%%%%%%%%%

\section{Introducción y Teoria}
\subsection{Introducción}
En el presente trabajo utilizaremos redes neuronales artificiales entrenadas con retropropagación de errores (backpropagation) para modelar dos problemas distintos. 

La idea será entrenar la red con información contenida en dos bases de datos.

El primer problema consiste en, dada unas muestras de tumores de mamas, poder calificar si un tumor corresponde o no con tumores \textbf{malignos} o \textbf{benignos}.

El segundo problema dada una muestra con ciertos atributos de edificios, tales como superficie total u orientación, podremos predecir la \textbf{cantidad de energía necesaria para calefaccionar y refrigerar los edificios}.

Se espera entonces, que utilizando lo que vimos en la materia podamos desarrollar redes que resuelvan correctamente ambos problemas. 

Adicionalmente, se espera hacer un análisis y experimentación del funcionamiento de un perceptron multicapa con retropropagación de errores.

\subsection{Introducción Teórica}
La idea principal del perceptron multicapa es utilizar el paradigma de aprendizaje supervisado con un algoritmo de corrección de errores. Es una red neuronal artificial formada por múltiples capas, esto le permite resolver problemas que no son linealmente separables, lo cual es la principal limitación del perceptron simple.

\begin{center}
\includegraphics[width=0.25\textwidth]{img/psimple}
\end{center}

El aprendizaje supervisado se basa en un entrenamiento en el cual se provee al sistema con información de las entradas y de igual forma se proveen las salidas esperadas para cada entrada en particular.

Intuitivamente, el perceptron multicapa permite aproximar funciones, categorizar y encontrar patrones.

\begin{center}
\includegraphics{img/pmulticapa}
\includegraphics[width=0.35\textwidth]{img/pmulticapa2}
\end{center}

Las neuronas de la capa oculta usan como regla de propagacion la suma ponderada de las entradas con los pesos sinápticos $w_{ij}$ y sobre esa suma ponderada, se aplica una función de transferencia de tipo sigmoide, que es acotada en respuesta.

\newpage
Gráfico de función sigmoide:

\begin{center}
\includegraphics[width=0.30\textwidth]{img/sigmoide}
\end{center}

Sobre el aprendizaje, para éste se suele usar en este tipo de redes recibe el nombre de backpropagation. Como función de coste global, se usa el error cuadrático medio. Es decir, que dado un par ($x_k$, $d_k$) correspondiente a la entrada k de los datos de entrenamiento y salida deseada asociada se calcula la cantidad.

Por otro lado, las capas pueden clasificarse en tres tipos:

\begin{enumerate}
\item \textbf{Capa de entrada:} Constituida por aquellas neuronas que introducen los patrones de entrada en la red. 
\item \textbf{Capas ocultas:} Formada por aquellas neuronas cuyas entradas provienen de capas anteriores y cuyas salidas pasan a neuronas de capas posteriores.
\item \textbf{Capa de salida:} Neuronas cuyos valores de salida se corresponden con las salidas de toda la red.
\end{enumerate}


En general, la estructura de una neurona se puede representar de la siguiente manera:

\begin{center}
\includegraphics[width=0.6\textwidth]{img/neurona}
\end{center}

Podemos observar a los elementos más importantes, al vector de entrada de la neurona $(x_1,...,x_m)$, a los pesos correspondientes como $w_{ij}$, a la función de activacio y al elemento de salida. A partir de esta estructura básica, la neurona puede mapear las entradas para obtener a la salida una respuesta deseada que pudiera pertenecer a alguna función a de terminar.


La función de activación trata de simular el mecanismo que realiza el sistema de neuronas en el cerebro, que se basa en la exitación de las neuronas hasta un cierto punto en el cúal se pasa un umbral en el que dicha neurona dispara la información que le corresponde. 

Entre las diferentes funciones de activación podemos encontrar la sigmoide.


Por otro lado, el perceptron cuenta con un coeficiente de entrenamiento que indica que tanto varian los pesos entre iteración e iteración, por lo cúal, indica que tan lento o rápido la red se entrena.
\section{Detalles implementativos}
Lorem ipsum dolor sit amet, consectetur adipisicing elit, sed do eiusmod
tempor incididunt ut labore et dolore magna aliqua. Ut enim ad minim veniam,
quis nostrud exercitation ullamco laboris nisi ut aliquip ex ea commodo
consequat. Duis aute irure dolor in reprehenderit in voluptate velit esse
cillum dolore eu fugiat nulla pariatur. Excepteur sint occaecat cupidatat non
proident, sunt in culpa qui officia deserunt mollit anim id est laborum.

\subsection{Algoritmos}
En esta sección describimos los pseudo-códigos de los algoritmos que utilizamos para resolver ambos ejercicios.

\begin{center}
\noindent\fbox{
\begin{minipage}{0.5\textwidth}
\begin{algorithm}[H]
 def train(): \\
  \While{not Fin}{
     	\For{$x \in D$}{
    		$y = X * W $\;
	 	\For{$j \in \{1..n\}$}{
			\For{$j \in \{1..n\}$}{
				$\widetilde{X}_{i}$ = $\emptyset$
				\For{$j \in \{1..n\}$}{
					$\widetilde{X}_{i}$ += $Y_k - W_{ik}$
				}
				$\Delta W_{ij} = \eta (X_i - \widetilde{X}_{i}) Y_j$
			}
		}
		$W += \Delta W_{ij} $
  	}
  }
 \caption{Train}
\end{algorithm}
\end{minipage}
}
\end{center}



\begin{center}
\noindent\fbox{
\begin{minipage}{0.5\textwidth}
\begin{algorithm}[H]
def activation(X):\\
$\widetilde{Y} = ||X^t - W||_2$\\
$Y = (\widetilde{Y} == min(\widetilde{Y}))$\\
$return Y$\\

 \caption{Activation}
\end{algorithm}
\end{minipage}
}
\end{center}


\begin{center}
\noindent\fbox{
\begin{minipage}{0.5\textwidth}
\begin{algorithm}[H]
def correction(X):\\
$j^* = nonzero(Y)$\\
$D = <\Lambda (j,j^*) | 0 \leq j \leq m >$\\
$\Delta W = \eta D(X_t - W)$\\
$W += \Delta W $\\
 \caption{Correction}
\end{algorithm}
\end{minipage}
}
\end{center}


\begin{center}
\noindent\fbox{
\begin{minipage}{0.5\textwidth}
\begin{algorithm}[H]
 def trainOjaMatricial(): \\
  \While{not Fin}{
     	\For{$x \in D$}{
    		$y = X * W $\;
	 	$\widetilde{X} = Y * W^T$\\

		$\Delta W= \eta (X - \widetilde{X}) Y$
  	}
  }
\caption{Train Oja Matricial}
\end{algorithm}
\end{minipage}
}
\end{center}


\begin{center}
\noindent\fbox{
\begin{minipage}{0.5\textwidth}
\begin{algorithm}[H]
 def trainSangerMatricial(): \\
  \While{not Fin}{
     	\For{$x \in D$}{
    		$U = triup(ones(M,M)) $\;
	 	$\widetilde{X} = W * (Y^t * U)$\\
		$\Delta W= \eta (X - \widetilde{X}) Y$
  	}
  }
\caption{Train Sanger Matricial}
\end{algorithm}
\end{minipage}
}
\end{center}




\section{Ejercicio: Cancer de mamas}
Los datos son resultado de un examen que se realizan para diagnosticar cancer de mamas. Para cada paciente contamos con 10 caracteristicas de las imagenes de las celulas, entre estas caracteristicas contamos con diagnostico, radio, perimetro, textura, area, suavidad, etc. 

Cada una de estas entradas corrresponde a los datos de distintos pacientes, es importante remarcar que se cuenta con el atributo del diagnostico final quen indica si un tumor es benigno o maligno (0 y 1 respectivamente).

Sobre este primer dataset se buscara entrenar una red neuronal para poder predecir, dado un nuevo paciente con sus datos, si su estado sera el de tener cancer o no.

Este tipo de problemas se denomina un \textbf{problema de clasificacion}

\section{Ejercicio: Eficiencia energetica}

En el siguiente problema contamos con los datos sobre la carga energetica para calefaccionar y refrigerar edificios en funcion de ciertas caractersisticas de los mismos.

El analisis se realizo tomando en cuenta edificios de diversas caracteristifcas, que difieren con respecto a la superficie y distribucion de las areas de reflejo, la orientacioon, etc.

Para cada edificio contamos con area de la superficie total, area de las paredes, area del techo, orientacion, area de reflejo total, etc.

Es importante remarcar que tambien contamos, para cada edifico del dataset, la cantidad de energio necersaria para realizar una calefaccion y refrigeracion adecuada.

Sobre este segundo dataset se buscara entrenar una red neuronal para poder aproximar la funcion que determina la cantidad de energia necesaria para
la calefaccion y refrigeracion de un edificio en funcion de los parametros de entrada. Es decir, es un problema de regresion.


\section{Experimentación}
\subsection{Búsqueda exhautiva}
Para comenzar la experimentación y tener una idea aproximada sobre los mejores valores de los parámetros del modelo utilizamos la técnica de \textit{grid search} o búsqueda exhaustiva. Siempre corrimos con un \textit{holdout set} de validación del 50\% del dataset completo, para poder simular la performance de nuestra implementación con datos nuevos que no se usaron en el entrenamiento. 

En el ejercicio uno, la búsqueda trata de minimar la suma de errores totales del set de validación, mientras que en el ejercicio dos se hace lo mismo pero al ser un problema de regresión, se define como error cuando el output de la red y el target es mayor a un \textit{epsilon} definido. 

La implementación es bastante simple, lo que hacemos es pasar por parámetro una lista con todos los posibles para parámetros y el algorítmo corre un ciclo, donde en cada ciclo prueba con un parámetro distinto guardandose el mejor modelo para finalmente imprimirlo por la consola. 

Además, se tuvo en cuenta el componente de inicialización aleatorio de la red neuronal. Para esto, decidimos correr 3 veces cada set de parámetros y quedarse con el mejor. 

Los parámetros que variamos fueron los siguientes:
\begin{itemize}
\item \textbf{Learning Rate}
\item \textbf{Momentum}
\item \textbf{Incremental o Batch}
\item \textbf{Configuración/Cantidad de capas y neuronas}
\end{itemize}

Una vez obtenidos los mejores parámetros, continuamos los experimentos variándolos a mano y analizando los resultados. Esto se explicará en la sección siguiente.

El algoritmo es el siguiente:

\begin{lstlisting}[caption=grid\_search]
def grid_search(param_grid):
	grid = ParameterGrid(param_grid)
	best_error = None
	best_params = None
	print "Ejercicio "+str(param_grid["1"])
	print len(grid), "modelos distintos"
	i = 0
	print "i 	error(training, validacion)"
	start = timer()
	for params in grid:
	    e = train(params['1'], params['2'], params['3'],params['4'],params['5'],params['6'],params['7'],
	    	params['8'],params['9'],params['10'],params['11'], True )
	    print i, "	", e
	    i += 1
	    if i%50 == 0:
	    	end = timer()
	    	print int((end-start)/60), " min"
	    if best_error == None or e < best_error:
	    	best_error = e
	    	best_params = [params['1'], params['2'], params['3'],params['4'],params['5'],params['6'],params['7'],params['8'],
	    	params['9'],params['10'],params['11']]
	print "MEJOR ERROR Y MODELO EJERCICIO "+str(param_grid["1"])
	print "training, validacion:", best_error
	print "parametros:", best_params
	end = timer()
	print int((end-start)/60), " min"
\end{lstlisting}

\newpage

Y la forma de instanciarlo es la siguiente:

\begin{lstlisting}[caption=Instanciación]
param_grid = {'1': [1],"2":['./datasets/tp1_ej1_training.csv'],"3": [None], "4":[0.1], "5": [200], 
			"6":[0.001,0.01,0.1,0.5, 0.005, 0.2, 0.3, 0.4], "7":[0.1,0.3,0.5, 0.7, 0.9], "8": [0.70], 
			"9": [0,1], "10":[0], "11":[[5],[10],[15],[20],[25],[15,15],[5,5],[10,10]]}
\end{lstlisting}


\subsection{Ejercicio 1}

Observamos que agregar capas no garantiza la convergencia y, cuando lo hace, provoca que las salidas oscilen abruptamente y tarde más épocas, por lo que optamos por utilizar \textbf{dos capas ocultas} como compromiso entre tiempo y calidad de solución. 

En cuanto al número de unidades por capa, ejecutamos el grid\_search 10 veces y en total encontramos que una buena solución es:

\begin{enumerate}
\item epsilon: 0.1
\item tau: 200
\item etha: 0.01
\item momentum: 0.9
\item holdoutRate: 0.7
\item modo: Incremental
\item unidades por capa: 5 y 5
\end{enumerate}

Y para esta obtuvimos los siguientes graficos.


\includegraphics[scale=0.4]{img/ej100109155sum}
\includegraphics[scale=0.4]{img/ej100109155mean}

En estos gráficos podemos ver que en el promedio, al promediar la cantidad de casos se puede ver que el trainning da mejor que validación. Al mismo tiempo, en la suma de errores la validación da menor porque hay menos casos de entrada. 
Por otro lado, vemos que si mantenemos la cantidad de épocas podemos caer en overfitting porque el error de trainning podría seguir disminuyendo pero el de validación no.

Un detalle importante es que esta arquitectura, cuando convergue, lo hace rápido. Como se pueder ver en este ejemplo de muestra que converge en un poco mas de 20 épocas.

De todas formas, analizando los gráficos tras varias corridas vimos que no siempre converge esta solución.
Como el campo de búsqueda es pequeño, probamos múltiples veces diferentes configuraciones de neuronas con nímeros de 2 a 15 neuronas por capa para tener una evidencia estadística de que sucede al variar este parámetro. Con ello, corroboramos que la solución arrojada por el algoritmo de búsqueda estaba cerca de un resultado optimo al ver las variaciones de diferentes medidas de error. 

Las medidas de error que tomamos fue el número de unidades con error, el error total, el promedio de error, condición de terminación de aprendizaje, es decir, si termino por error menor a epsilon o por límite de épocas. En particular, por la naturaleza del problema, nos interesa ver la Precisión y el Recall, o dicho de otra manera, el índice de falso positivo y falso negativos. 

Como medida utilizaremos la Media armónica donde 

$Media armonica=2*precision*recall/(precision+recall)$

$precision=true positive/(true positive+false positive)$

$recall= true positive/(true positive+false negative)$


Para determinar el valor de la primer capa, elegimos aquellas configuraciones que nunca terminen debido al límite de épocas. Luego, en base a la cantidad de unidades con error y el error total cometido, las clasificamos en este orden. 
En particular, obtuvimos que las configuraciones de [8,7] y [9,5] arrojan buenos resultados (en base a la suma de errores y el promedio).

Podemos ver a continuación algunos gráficos obtenidos variando los parámetros con 8 y 7 capas.

\includegraphics[scale=0.4]{img/ej100207187sum}
\includegraphics[scale=0.4]{img/ej100207187mean}

En particular este gráfico fue el mejor de los gráficos que obtuvimos variando los parámetros para esta cantidad de capas. Y los parámetros son:
\begin{enumerate}
\item epsilon: 0.1
\item tau: 200
\item etha: 0.02
\item momentum: 0.7
\item holdoutRate: 0.7
\item modo: Incremental
\item unidades por capa: 8 y 7
\end{enumerate}

Se puede ver en estos gráficos que a simple vista, el promedio de errores tiene un mayor número de oscilaciones. También vemos, que comparando a los gráficos del [5,5], este converge en una cantidad de épocas mayor.

\newpage 

Podemos ver a continuación algunos gráficos obtenimos variando los parámetros con 9 y 5 capas.

\includegraphics[scale=0.4]{img/ej100505195sum}
\includegraphics[scale=0.4]{img/ej100505195mean}

En particular, este gráfico fue el mejor de los gráficos que obtuvimos variando los parámetros para esta cantdiad de capas. Y los parámetros son:
\begin{enumerate}
\item epsilon: 0.1
\item tau: 200
\item etha: 0.05
\item momentum: 0.5
\item holdoutRate: 0.7
\item modo: Incremental
\item unidades por capa: 9 y 5
\end{enumerate}

Se puede ver en estos gráficos que el promedio de errores es bastante variado y que al igual que con [8,7] la cantidad de épocas es mayor que [5,5].

Notamos que el número de neuronas en la última capa permite que la salida se ajuste al resultado mientras que la primer capa realiza el grueso del esfuerzo de generalización. Al contrario de lo que esperabamos, incrementar el número de unidades de la última capa no mejora el resultado, si no que lo empeora debido a que le lleva un mayor número de épocas para converger (cuando lo hace). Pocas neuronas en la capa final mostraron experimentalmente que proveen una buena solución.

Por lo mencionado anteriormente, analizamos el valor del Mean Armonic para una muestra que reservamos del dataset original (que no se uso en el entrenamiento) y obtuvimos que el caso de [8,7] capas da un valor del 0.94527, el caso de [9,5] capas da un valor del 0.9508 y el caso de [5,5] capas da un valor de 0.97.

En conclusión, decidimos elegir como modelo el de 5 y 5 capas dado el Mean Armonic da más cercano a 1, es un modelo con pocas capas entonces tiene una velocidad de procesamiento mayor.

\newpage

Además, analizamos los histogramas para ver la cantidad de equivocaciones a la hora de predecir si un cáncer es benigno o maligno obteniendo para el caso de [5,5]:

\begin{center}
\includegraphics[scale=0.4]{img/histogramaej155}
\end{center}

y para el caso de [9,5] el siguiente gráfico

\begin{center}
\includegraphics[scale=0.4]{img/ej195_histograma}
\end{center}

Concluyendo, podemos ver que en el gráfico de [5,5] capas la cantidad de equivocaciones es 8, siendo este un número menor que el de [9,5] capas. Tener en cuenta que la cantidad de muestras es de 400.

\newpage

\subsection{Ejercicio 2}

Al igual que en el ejercicio 1, en este ejercicio se corrió el algoritmo de grid\_search 10 veces y en total encontramos dos  soluciones buenas. 

La primera:

\begin{enumerate}
\item epsilon: 0.01
\item tau: 200
\item etha: 0.005
\item momentum: 0.9
\item holdoutRate: 0.7
\item modo: Incremental
\item unidades por capa: 5
\end{enumerate}

Y para esta, obtuvimos los siguientes gráficos.

\includegraphics[scale=0.4]{img/ej200050915sum}
\includegraphics[scale=0.4]{img/ej200050915mean}

Se puede ver en estos gráficos que en el gráfico de suma de errores las dos curvas se comportan igual, esto es porque es un problema de regresión y las funciones utilizadas son linealmente dependiente, es decir, la salida es proporcional a la entrada.

Un detalle importante es que en este gráfico vemos como la curva \textbf{no se plancha}, y este era un problema que teniamos en la mayoria de los casos de 2 capas. 

La otra red que nos arrojo el grid\_search es la siguiente:

\begin{enumerate}
\item epsilon: 0.01
\item tau: 1000
\item etha: 0.01
\item momentum: 0.6
\item holdoutRate: 0.85
\item modo: Incremental
\item unidades por capa: 8 y 5
\end{enumerate}

\newpage

Con esta arquitecutura generamos las siguientes imágenes:

\includegraphics[scale=0.4]{img/asum}
\includegraphics[scale=0.4]{img/bmean}

Se puede ver en estos gráficos que la suma de errores se mantiene constante en un intervalo, es decir, parece que tiene una asintota, en otras palabras, \textbf{se plancha}. Esto es un error típico encontrado en las ejecuciones de 2 capas. Creemos que esto es debido a la función de activación dado que, utilizamos una sidmoide y entonces ésta tiene este comportamiento.

Por otra parte, al igual que en el ejercicio anterior, probamos a mano cambiar los valores de los parámetros pero de todas formas no encontramos nada mejor. En general, todos los resultados que obtuvimos fueron similar a este:

\includegraphics[scale=0.4]{img/ccsum}
\includegraphics[scale=0.4]{img/bbmean}

estos gráficos corresponden a los siguientes parámetros:

\begin{enumerate}
\item epsilon: 0.01
\item tau: 200
\item etha: 0.1
\item momentum: 0.9
\item holdoutRate: 0.7
\item modo: Batch
\item unidades por capa: 15
\end{enumerate}
En estos gráficos se puede ver una notable inestabilidad por parte de la red (se ve en ambos gráficos). También podemos ver que, la suma de errores se mueve en un intervalo fijo, es decir, no esta descendiendo. 

\newpage

Como se podrá ver en los gráficos elegimos el primer modelo porque es el que mas desciende y también tiene menor cantidad de capas, por lo tanto procesamiento más rapido. También vemos que la derivada de la función de promedio de errores nunca es cero, esto quiere decir que el modelo aprende.

Para este modelo elegido, hicimos unos gráficos de disperción. 

En este gráfico se puede ver en el eje \textbf{y} el valor predicho y en el eje \textbf{x} el valor objetivo. En general lo ideal es concluir en un grafico como el siguiente:

\begin{center}
\includegraphics[scale=0.4]{img/regresionperfecta}
\end{center}

Con nuestra arquitectura obtuvimos los siguientes gráficos:

\includegraphics[scale=0.4]{img/ej2-calef}
\includegraphics[scale=0.4]{img/ej2-refrig}


En estos gráficos notamos que son similares a lo ideal, sin caer en un overfitting, los puntos obtenidos se acercan a la recta Y=X, lo que quiere decir que los valores aprendidos se acercan mucho al objetivo.

Como se explicó en las secciones anteriores el output fue normalizado, por esto la escala del gráfico va desde -1 a 1 que es el rango de la función sigmoide bipolar.

\section{Entregable}
\subsection{Contenido del entregable}

Dentro del archivo comprimido entregado se encuentra los archivos python correspondientes a la implementación del trabajo práctico. También se encuentra el informe y las redes entrenadas del ejercicio 1 y del ejercicio dos (red\_ej1 y red\_ej2 respectivamente).

Para la ejecuci\'on se necesitan las librerias matplotlib,sklearn y numpy.
En caso de no tener alguna, ejecutar por consola el siguiente comando.

\begin{verbatim}
$ sudo apt-get install python-numpy python-sklearn python-matplotlib
\end{verbatim}

\subsection{Modo de ejecución}

La manera de ejecutar el programa dado es la siguiente:

Por default, los parámetros a partir de epsilon inclusive son opcionales, se entrena con parámetros seteados.

%python main.py ej1 -t './datasets/tp1_ej1_training.csv' 'red' 0.1 10 0.75  0.01 0.6 1 0 
\begin{verbatim}
$ python main.py < ejercicio > -t < archivo_entrada > < archivo_red_salida >  < epsilon > 
                       < tau > < learning_rate > < momentum > < holdout_rate > < modo > <capas_ocultas> 
\end{verbatim}
\begin{verbatim}
$ python main.py < ejercicio > -l < archivo_entrada > < archivo_red_salida > 
\end{verbatim}

Donde:

\begin{enumerate}
\item ejercicio: \textbf{ej1} o \textbf{ej2} dependiendo el ejercicio que se quiera ejecutar.
\item opción: 
\begin{enumerate}
\item -t: Entrena y guarda la red.
\item -l: Carga la red ya entrenada y prueba el nuevo dataset.
\end{enumerate}
\item archivo\_entrada: archivo de dataset.
\item archivo\_red\_salida: archivo donde se guardara la red.
\item epsilon: Error maximo permitido.
\item tau: Numero maximo de epocas para ejecutar.
\item learning\_rate: Coeficiente de aprendizaje.
\item momentum: Momentum
\item holdout\_rate: Porcentaje del dataset con el que se va a entrenar.
\item modo: 
\begin{enumerate}
\item 1: Incremental
\item 0: Batch
\end{enumerate}
\item capas\_ocultas: Numero de unidades en cada capa oculta.
\end{enumerate}

Por ejemplo: 

Si queremos ejecutar el ejercicio 1 entrenandolo con el dataset \textbf{tp1\_ej1\_training.csv}, guardando la red en el archivo 
\textbf{red} con un epsilon de \textbf{0.1}, tau de \textbf{1000}, holdout\_rate de \textbf{0.7} learning\_rate de \textbf{0.01}, momentum de \textbf{0.9} con el modo \textbf{Incremental} y 10 neuronas en la primer capa y 2 en la segunda seria:

\begin{verbatim}
$ python main.py ej1 -t 'tp1_ej1_training.csv' ’red’ 0.1 1000 0.01 0.9 0.7 1 [10,2]
\end{verbatim}


En caso de querer cargar una red ya guardada seria, por ejemplo:

\begin{verbatim}
$ python main.py ej1 -l 'red' 'tp1_ej1_training.csv'
\end{verbatim}

Siendo \textbf{red} la red guardada y \textbf{tp1\_ej1\_training.csv} con el que va a probar.
\section{Codigos}
\subsection{main.py}
\emph{\color{red} ACTUALIZAR }
\begin{lstlisting}[caption=main.py]
from new_perceptron import perceptron_Multiple,sigmoidea_bipolar,sigmoidea_logistica
from numpy import *
from sklearn.grid_search import ParameterGrid
from pylab import Line2D, plot, axis, show, pcolor, colorbar, bone, savefig
from timeit import default_timer as timer 

import pylab as plt
import sys
import cPickle
import datetime

def imprimirImagen(ejercicio, error_t_hist,error_v_hist, suma, img_name, prnt=False, save=True):
	plt.xlabel('Epoch')
	plt.title("Ejercicio "+str(ejercicio))
	plt.plot(error_t_hist, label="Training")
	if suma:
		img_name += "_sum.png"
		plt.ylabel('Suma de errores')
	else:
		img_name += "_mean.png"
		plt.ylabel('Promedio de errores')
	plt.plot(error_v_hist, label="Validacion")
	plt.grid()
	plt.legend()
	if prnt:
		plt.show()
	if save:
		plt.savefig(img_name)
	plt.close()

def train(ejercicio,Dataset=None, save_file=None, epsilon=0.1, tau=1000, etha=0.01,m=0.6,holdoutRate=0.5, modo=0, 
	early=0,unitsPorCapa=[15], wrapper=False):
	if not wrapper:			
		print "Ejercicio ",ejercicio	
		print "File "+str(Dataset)
		print "Unidades por capa " + str(unitsPorCapa)
		print "Error aceptable " + str(epsilon)
		print "Max Epocas " +str(tau)
		print "Learning Rate " + str(etha)
	Red=perceptron_Multiple(unitsPorCapa,epsilon,tau,etha,m,holdoutRate)
	Red.load_dataset(Dataset, ejercicio)
	i = 1
	best_error = -1
	if wrapper:
		error_t_hist_best = None
		error_v_hist_best = None
		error_v_hist_sum_best = None
		error_t_hist_sum_best = None
		i = 3
	for _ in xrange(i):
		error_t_hist,error_v_hist, error_v_hist_sum, error_t_hist_sum, epoch = Red.train(modo, early)
		if best_error == -1 or (wrapper and error_v_hist_sum[-1] < best_error):
			best_error = error_v_hist_sum[-1]
			error_t_hist_best = error_t_hist
			error_v_hist_best = error_v_hist
			error_v_hist_sum_best = error_v_hist_sum
			error_t_hist_sum_best = error_t_hist_sum
	if not wrapper:
		print '>> epochAlcanzada: ' + str(epoch)
		print '>> error promedio Training:	' + str(error_t_hist_best[-1])
		print '>> error promedio Testing:	' + str(error_v_hist_best[-1])
		print '>> suma de errores Training:	' + str(error_t_hist_sum_best[-1])
		print '>> suma de errores Testing:	' + str(error_v_hist_sum_best[-1])	
	img_name= "ej"+str(ejercicio)+"_"+str(etha)+"_"+str(m)+"_"+str(modo)+"_"+str(unitsPorCapa)	
	imprimirImagen(ejercicio, error_t_hist_best,error_v_hist_best, suma=False, img_name=img_name)
	imprimirImagen(ejercicio, error_v_hist_sum_best, error_t_hist_sum_best, suma=True, img_name=img_name)
	
	if(save_file!=None):
		print "Guardando Red"
		with open(save_file, "wb") as output:
			cPickle.dump(Red, output, cPickle.HIGHEST_PROTOCOL)
	return error_t_hist_sum[-1], error_v_hist_sum[-1]

def load(ej,Net,Dataset, prnt=True):
	print "Cargando Red tipo Ejercicio ",ej
	with open(Net, "rb") as input:
		Red = cPickle.load(input) # protocol version is auto detected
	Red.load_dataset(Dataset, ej)
	lista_error,errorTotal,cantidad_errores = Red.evaluate()
	print '>> error total acumulado: ' + str(errorTotal)+' Numero de equivocaciones: '+str(cantidad_errores)	
	plt.plot(lista_error)
	plt.ylabel('Valor del error')
	if prnt:
		plt.show()
	return errorTotal, cantidad_errores	

def grid_search(param_grid):
	grid = ParameterGrid(param_grid)
	best_error = None
	best_params = None
	print "Ejercicio "+str(param_grid["1"])
	print len(grid), "modelos distintos"
	i = 0
	print "i 	error(training, validacion)"
	start = timer()
	for params in grid:
	    e = train(params['1'], params['2'], params['3'],params['4'],params['5'],params['6'],params['7'],params['8'],
	    	params['9'], params['10'],params['11'], True )
	    print i, "	", e
	    i += 1
	    if i%50 == 0:
	    	end = timer()
	    	print int((end-start)/60), " min"
	    if best_error == None or e < best_error:
	    	best_error = e
	    	best_params = [params['1'], params['2'], params['3'],params['4'],params['5'],params['6'],params['7'],params['8'],
	    	params['9'], 	params['10'],params['11']]
	print "MEJOR ERROR Y MODELO EJERCICIO "+str(param_grid["1"])
	print "training, validacion:", best_error
	print "parametros:", best_params
	end = timer()
	print int((end-start)/60), " min"
	
args = sys.argv
message = "\nModo de uso:\n\
python main.py (ej1|ej2) -t nomDataSet nomFfileout parametros\n\
Con los siguientes parametros en orden:\n\
epsilon tau etha momentum holdoutRate modo_aprendizaje\n\n\
-t es para entrenar\
-l es para testear\
modo_aprendizaje = 0 para batch 1 para incremental\
"

# print len(args)
if len(args)==1:
	pruebas()
	sys.exit()
elif len(args) < 5:
	print message
	sys.exit()

cmdTrain = args[2] == "-t"
cmdLoad = args[2] == "-l"

if(args[1] == "ej1"):
	ejercicio=1
elif(args[1] == "ej2"):
	ejercicio=2
else:
	print message
	sys.exit()

if cmdTrain:
	
	if len(args) < 3:
		print "\nIncorrecta cantidad de argumentos para entrenar."
		print message
		sys.exit()
	archivoDataset = args[3]
	archivoRed=None
	errorAceptable=0.1
	maxEpoch=1000
	learningRate=0.01
	momentum=0.6
	holdoutRate=0.5
	modo=0
	early=0
	unitsPorCapa=[5]
	if len(args) > 4:
		archivoRed=args[4]
	if len(args) > 5:
		errorAceptable = float(args[5])
	if len(args) > 6:
		maxEpoch = int(args[6])
	if len(args) > 7:
		learningRate = float(args[7])
	if len(args) > 8:
		momentum = float(args[8])
	if len(args) > 9:
		holdoutRate = float(args[9])
	if len(args) > 10:
		modo=int(args[10])
	if len(args) > 11:
		early=int(args[11])
	if len(args) > 12:
		#unitsPorCapa=int(args[11])
		str1=str(args[12])
		unitsPorCapa=map(int, str1[1:-1].split(','))
	train(ejercicio,archivoDataset,archivoRed,errorAceptable,maxEpoch,learningRate,momentum,holdoutRate,modo, early,
	 unitsPorCapa)

elif cmdLoad:
	
	if len(args) != 5:
		print "\nIncorrecta cantidad de argumentos para entrenar."
		print message
		sys.exit()
		
	archivoRed = args[3]
	archivoDataset = args[4]
	
	load(ejercicio,archivoRed, archivoDataset)
else:
	print message
\end{lstlisting}


\subsection{new\_perceptron.py}

\begin{lstlisting}[caption=new\_perceptron.py]
from numpy import *
import math
def sigmoidea_bipolar(vector, derivative=False):
	B = 1
	if derivative:
		return B * (1 -sigmoidea_bipolar(vector)**2)
	else:
		return tanh(B*vector)

def sigmoidea_logistica(vector, derivative=False):
	B = 1
	if derivative:
		return B * (1 -sigmoidea_logistica(vector))*sigmoidea_logistica(vector)
	else:
		return 1/(1+exp(-vector))

class perceptron_Multiple:

	def load_dataset(self,dataset, ejercicio):
		# print "> Cargando dataset..."	
		f = open(dataset)
		self.X = []
		self.Z = []
		for line in f:
			if line.rstrip():
				if ejercicio == 1:
					r = line.rstrip().split(", ")
					x_i = map(float, r[1:])
					z_i = map(self.cod, r[0])
				else:
					r = line.rstrip().split(",	")
					x_i = map(float, r[:-2])
					z_i = map(float, r[-2:])
				self.X.append(x_i)
				self.Z.append(z_i)
		self.normalizar(self.X)
		if ejercicio != 1:
			self.normalizar(self.Z)
		self.X = array(self.X) # Para pretty print
		self.Z = array(self.Z)

	def __init__(self,UnitsXCapa=[15],e=0,t=0,nl=0,m=0.6,holdout=1,funcionActivacion=sigmoidea_bipolar):
		self.funcActivacion = funcionActivacion	
		self.epsilon = e
		self.tau = t
		self.eta = nl
		self.p = holdout
		self.momentum = m
	 	# CANT CAPAS
	 	self.L=len(UnitsXCapa)+2
	 	self.UnitsXCapa=UnitsXCapa
	 	self.Beta=1

	def cod(self,c):
		if c == "M":
			return 1
		else:
			return -1 

	def normalizar(self, array):
		media = mean(array, axis= 0) 
		varianza = std(array, axis=0)
		for i in xrange(len(array)):
			array[i] = (array[i] - media )/varianza	

	def train(self,modo=0, early=0):
		self.P = len(self.X)
		self.S = [shape(self.X)[1]]
		self.S.extend(self.UnitsXCapa)
		self.S.append(shape(self.Z)[1])
		self.S = array(self.S)
		# TAMANOS W, dW, Y
		# Basura en la pos 0, indizado desde 1
		self.W = array([random.uniform(-sqrt(self.S[j]),sqrt(self.S[j]), (self.S[j-1]+1, self.S[j])) for j in range(self.L)])
		# Basura en la pos 0, indizado desde 1
		self.dW = array([zeros((self.S[j-1]+1, self.S[j])) for j in range(self.L)])
		self.Y = [zeros((1, self.S[j]+1)) for j in range(self.L-1)]
		self.Y.append([zeros((1,shape(self.Z)[1]))])
		t,error_v_hist,error_t_hist, error_t_hist_sum, error_v_hist_sum = self.holdout(self.epsilon, self.tau, self.p, modo, early)
		# print error_t_hist[-1],error_v_hist[-1] ,t 
		return error_t_hist,error_v_hist, error_v_hist_sum, error_t_hist_sum, t

	def holdout(self,epsilon, tau, p, modo, early):
		e_t = 1
		e_v = 1
		t = 0
		v = int(p*self.P)
		error_v_hist=[]
		error_t_hist=[]
		error_v_hist_sum=[]
		error_t_hist_sum=[]
		early_count = 0
		while(t<self.tau and e_t > self.epsilon):
			# if t == 0 and modo:
			# 	print "Modo incremental"
			# elif t == 0 and not modo:
			# 	print "Modo batch"			
			e_t, e_t_sum = self.training(self.X[:v],self.Z[:v], modo)
			e_v, e_v_sum = self.testing(self.X[v:],self.Z[v:])
			error_v_hist.append(e_v)
			error_t_hist.append(e_t)
			error_v_hist_sum.append(e_v_sum)
			error_t_hist_sum.append(e_t_sum)
			t += 1
			#early_count = (early_count+1) if t > 2 and error_v_hist[-1]>error_v_hist[-2] else 0
			# if(t % 10==1):
			# 	print "epoch", t, "   e_training", e_t, "	e_validation", e_v
			# if early and early_count>=30:
			# 	print "Early Stopping - 30 epochs de crecimiento de error de validacion"
			# 	break
		return t,error_v_hist,error_t_hist, error_t_hist_sum, error_v_hist_sum

	def training(self,X, Z, modo):
		e_count = 0
		e_sum = 0
		for h in range(len(X)): 
			self.activation(X[h])
			e = self.correction(Z[h])
			if e > self.epsilon:
				e_count += 1
			e_sum += e
			if modo:
				self.adaptation()
		if not modo:
			self.adaptation()	
		return e_sum/(len(X) if len(X) != 0 else 1), e_count

	def activation(self,X_h):
		self.Y[0] = append(X_h, [-1])[newaxis]
		for j in range(1, self.L):
			if j == self.L-1:
				self.Y[j] = self.funcActivacion(dot(self.Y[j-1], self.W[j]))
			else:
				self.Y[j] = append(self.funcActivacion(dot(self.Y[j-1], self.W[j])), [-1])[newaxis]
		return self.Y[-1]

	def correction(self,Z_h):
		E = Z_h - self.Y[-1]
		e = (E**2).sum()
		for j in range(self.L-1, 0, -1):   
			D = E*self.funcActivacion(dot(self.Y[j-1], self.W[j]), True)
			self.dW[j] += (self.eta*dot(transpose(self.Y[j-1]), D)) 
			# El error no tiene sentido q tenga el -1 del final
			E = dot(D, transpose(self.W[j]))[0][:-1]
		return e

	def adaptation(self):
		for j in range(1, self.L):
			self.W[j] += self.dW[j] 
			self.dW[j] *= self.momentum

	def testing(self,X, Z):
		e_count = 0
		e_sum = 0
		for (X_h, Z_h) in zip(X, Z):
			E = self.activation(X_h)-Z_h
			e = (E**2).sum()
			if e > self.epsilon:
				e_count += 1
			e_sum += e
		return e_sum/(len(X) if len(X) != 0 else 1), e_count

	def evaluate(self):	
		e_sum = 0
		list_error=[]
		numero_error=0
		for (X_h, Z_h) in zip(self.X, self.Z):
			E=self.activation(X_h)-Z_h
			if((E**2).sum()>=1):
				numero_error+=1
			e_sum+=(E**2).sum()
			list_error.append((E**2).sum())
		return list_error,e_sum/(len(self.X) if len(self.X) != 0 else 1),numero_error	



\end{lstlisting}



\end{document}
