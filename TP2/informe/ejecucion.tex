\section{Entregable}
\subsection{Contenido del entregable}
Lorem ipsum dolor sit amet, consectetur adipisicing elit, sed do eiusmod
tempor incididunt ut labore et dolore magna aliqua. Ut enim ad minim veniam,
quis nostrud exercitation ullamco laboris nisi ut aliquip ex ea commodo
consequat. Duis aute irure dolor in reprehenderit in voluptate velit esse
cillum dolore eu fugiat nulla pariatur. Excepteur sint occaecat cupidatat non
proident, sunt in culpa qui officia deserunt mollit anim id est laborum.

\subsection{Modo de ejecución}




\subsubsection{Ejercicio 1}


La manera de ejecutar el programa dado es la siguiente:

En caso de querer entrenar la red sera:

\begin{verbatim}
$ python main.py < archivo_entrada > < archivo_red_salida > -train < lrate > 
                                                   < max_epochs > < method >    
\end{verbatim}

Donde:

\begin{enumerate}
\item archivo\_entrada: archivo de dataset.
\item archivo\_red\_salida: archivo donde se guardara la red.
\item lrate: Coeficiente de aprendizaje.
\item max\_epochs: Maxima cantidad de epocas permitidas.
\item method: 
\begin{enumerate}
\item -s: Sanger
\item -o: Oja
\end{enumerate}
\end{enumerate}

Por ejemplo: 

Si queremos ejecutar el ejercicio 1 entrenandolo con el dataset \textbf{tp2\_training\_dataset.csv}, guardando la red en el archivo 
\textbf{red} con un learning\_rate de \textbf{0.01}, con un espacio de salida de \textbf{3}, con una cantidad maxima de epocas de \textbf{10000} y con el modo \textbf{Sanger} seria:

\begin{verbatim}
$ python main.py tp2_training.dataset.csv  red -train 0.01 10000 -s
\end{verbatim}

En caso de querer cargar una red ya guardada seria:

\begin{verbatim}
$ python main.py < archivo_entrada > < nombre_red_in > -load
\end{verbatim}

Donde:

\begin{enumerate}
\item archivo\_entrada: archivo de dataset.
\item nombre\_red\_in: nombre de la red a cargar.
\end{enumerate}

Por ejemplo, si queremos cargar la red llamada \textbf{red} con el dataset \textbf{tp2\_training\_dataset.csv} seria:

\begin{verbatim}
$  python main.py tp2_training_dataset.csv red -load
\end{verbatim}

\subsubsection{Ejercicio 2}

La manera de ejecutar el programa dado es la siguiente:

En caso de querer entrenar la red sera:

\begin{verbatim}
$ python main.py < archivo_entrada> <archivo_red_salida> -train < sigmaInicial > 
                                   < lrateInicial > < dimX > < dimY > < epochs >
\end{verbatim}

Donde:

\begin{enumerate}
\item archivo\_entrada: archivo de dataset.
\item archivo\_red\_salida: archivo donde se guardara la red.
\item sigmaInicial: Signma inicial
\item epochs: Maxima cantidad de epocas permitidas.
\item lrateInicial: Learning Rate inicial
\item dimX: Dimension de X.
\item dimY: Dimension de Y.
\end{enumerate}

Por ejemplo: 

Si queremos ejecutar el ejercicio 2 entrenandolo con el dataset \textbf{tp2\_training\_dataset.csv}, guardando la red en el archivo \textbf{red} con un signma inicial de 1, un learning\_rate de \textbf{0.001}, con una cantidad de epocas de \textbf{1500}, con dimensiones X=Y=9 seria:

\begin{verbatim}
$ python main.py tp2_training_dataset.csv red -train 1 0.001 9 9 1500
\end{verbatim}

En caso de querer cargar una red ya guardada seria:

\begin{verbatim}
$ python main.py < archivo_entrada> <nombre_red_in> -load 
\end{verbatim}

Donde:

\begin{enumerate}
\item archivo\_entrada: archivo de dataset.
\item nombre\_red\_in: nombre de la red a cargar.
\end{enumerate}

Por ejemplo, si queremos cargar la red llamada \textbf{red} con el dataset \textbf{tp2\_training\_dataset.csv} seria:

\begin{verbatim}
$ python main.py tp2_training_dataset.csv red -load
\end{verbatim}

