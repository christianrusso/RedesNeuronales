\section{Conclusiones}
Para clasificar textos, ambas redes dieron resultados bastante buenos. Un factor importante a tener en cuenta es que las redes de Oja y Sanger son mucho mas rapidas para obtener un resultado que la red de kohonen pero son mas complicadas para analizar, dado que las soluciones viven en un espacio tridimensional. Aunque sus medidas no son comparables, se puede apreciar que comparten el mismo problema cuando una categoria se superpone caracteristicas con una o mas categorias distintas y se dificulta su clasificacion. Sin embargo, este problema se puede subsanar en kohonen simplemente al incrementar el espacio de salida, a cambio de mayor tiempo de procesamiento, en cambio, para Oja y Sanger esto no es posible.